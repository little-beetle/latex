\documentclass[12pt,fleqn]{article}

\usepackage{amsmath,amssymb,amsthm,enumerate,color,graphics,epsfig,url}

\usepackage{cmap}

\usepackage[pdftex,
colorlinks,%
linkcolor=blue,citecolor=red,urlcolor=blue,
hyperindex,%
plainpages=false,%
bookmarksopen,%
bookmarksnumbered,%
unicode]{hyperref}
\usepackage [dvipsnames] {xcolor}
\usepackage{pgf,tikz,pgfplots,tikz-3dplot}
\usetikzlibrary{calendar,folding}
\usetikzlibrary{arrows,patterns,decorations.pathmorphing,backgrounds,positioning,fit,petri}
\usetikzlibrary{calc,3d,intersections,shapes}
%\pgfplotsset{compat=1.11}
\usepgfplotslibrary{polar}

\usepackage[utf8]{inputenc}
\usepackage[ukrainian]{babel}

\setlength{\textwidth}{160.0mm}
\setlength{\textheight}{240.0mm}
\setlength{\oddsidemargin}{0mm}
\setlength{\evensidemargin}{0mm}
\setlength{\topmargin}{-18mm}
\setlength{\parindent}{5.0mm}

\newtheorem{theorem}{Теорема}[section]
\newtheorem*{theoremN}{Теорема}
\newtheorem{proposition}{Твердження}[section]
\newtheorem{statement}{Твердження}[section]
\newtheorem{lemma}{Лема}[section]
\newtheorem{corollary}{Наслідок}[section]

{\theoremstyle{definition}
\newtheorem{definition}{Означення}[section]
\newtheorem{example}{Приклад}[section]
\newtheorem{problem}{Задача}[section]
\newtheorem*{problem*}{Задача}
\newtheorem{question}{Питання}[section]
\newtheorem{exercise}{Вправа}[section]
}

\usepackage[labelsep=period]{caption}
\numberwithin{figure}{section}
\numberwithin{equation}{section}

\begin{document}

\large

\setcounter{page}{0}
\thispagestyle{empty}
\centerline{Національний технічний університет України}
\centerline{``Київський політехнічний інститут'' ім.~І.~Сікорського}
\centerline{фізико-математичний факультет}

\vspace{50mm}

\begin{center}
\Large\bf
Курсова робота\\[5mm]
З предмету: <<Теорія ймовірностей>>\\
Тема: <<Процес Пуассона>>

\end{center}

\vspace{50mm}

\begin{flushright}
\begin{minipage}{70mm}
\large Виконали студенти ФМФ:\\
групи ОМ-01 та ОМ-02\\
Герасимчук Олександра, \\
Лиховид Варвара, \\
Баршеніков Максим, \\
Ковальчук Володимир, \\
Шмат Вікторія
\end{minipage}
\end{flushright}
\newpage

\tableofcontents

%\listoffigures

\newpage

\section{Випадкові процеси в безперервному часі}\label{random processes}
\emph{Стисло} Це перелік випадкових процесів у безперервному часі зосереджених на Пуассонівському процесі. Зв'язок між Пуассонівським процесом і експоненціальним розподілом відкритим за допомогою властивості відсутності наслідків. Прості процеси народження та народження-смерті описано, після чого йде вступ до теорії масового обслуговування.

\subsection{Життя на телефонному щиті}\label{sub.11.1}
Розгалуджені процеси та випадкові блукання є двома прикладами випадкових процесів. Кожен є випадковою послідовністю, і ми називаємо їх процесам з дискретним часом, оскільки вони включають спостереження за дискретні часи $n=0,1,2\dots$. Багато інших процесів включають спостереження, які проводяться безпосередньо з плином часу, і такі процеси називають процесами безперервного часу. Швидше ніж бути множиною $(Z_n:n=0,1,2 \dots)$ випадкових величин, індексовиних невід'ємними цілими числами, безперервний процес є множиною $Z=(Z_t:t\geq 0)$ випадкових величин, індексованих континуумом $[0,\infty]$, де ми вважаємо $Z_t$ значенням процесу в момент часу $t$. Загальна теорія безперервних процесів досить глибока і досить складна, але більшу частину основних труднощів можна уникнути, якщо обмежити нашу увагу процесами, які приймають цілі значення тобто процеси, для $Z_t$ є (випадковим) цілим числом для кожного $t$, і всі наші приклади з цієї форми. Принципова відмінність між вивченням таких процесів безперервного часу і вивчення процесів з дискретним часом є лише тим, що виникає при переході від цілих чисел до дійсних чисел: замість встановлення рекурентних рівнянь і різницевих рівнянь (наприклад, (9.5) і (10.26)) встановимо диференціальні рівняння.

Ось наш базовий приклад. Білл є головним швейцаром у Гранд Готелі, а частиною його роботи є відповідати на вхідні телефонні дзвінки. Він не може передбачити з упевненістю, коли телефон буде дзвонити; з його точки зору, дзвінки надходять випадково. Робимо два спрощуючих припущення щодо цих дзвінків. По-перше, ми припускаємо, що Білл миттєво обробляє кожен дзвінок, так що жоден виклик не буде втрачено, якщо він не надійде точно в той самий момент, що й інший (на практиці Біллу потрібно підійти до телефону й поговорити, а це потребує часу --- складніші моделі потребують рахунок цього). По-друге, ми припускаємо, що дзвінки надходять <<однорідно>> за часом, у такому сенсі що ймовірність того, що телефон дзвонить протягом певного часу, залежить лише від тривалості цього періоду (звичайно, це абсурдне припущення, але воно може бути справедливим на певну частину дня). Ми описуємо час параметром $t$, який приймає значення в проміжку $[0,\infty]$, і пропонуємо наступну модель для надходження телефонних дзвінків на комутатор. Нехай $N_t$ представляє кількість дзвінків, які надійшли за часовий інтервал $[0,\infty]$: тобто $N_t$ кількість вхідних дзвінків, які Білл обробив до часу $t$ включно. Ми припускаємо, що випадковий процес $N=(N_t:t\geq 0)$ розвивається таким чином, що виконуються такі умови:
\begin{enumerate}
  \item $N_t$ --- це випадкова змінна, яка приймає значення в $\{0,1,2,\dots \}$,
  \item $N_0=0$,
  \item $N_s\leq N_t$, якщо $s\leq t$,
  \item \emph{незалежність}: якщо $0\leq s < t$, то кількість викликів, які находять протягом інтервалу часу $(s,t]$ не залежить від надходження викликів до часу $s$,
  \item \emph{швидкість прибуття}: існує число $\lambda (>0)$, яке називається швидкістю прибуття, таке \footnote{Згадайте позначення Ландау зі с. 127: $o(h)$ позначає деяку функцію від $h$, яка має менший порядок величини ніж $h$ при $h\rightarrow 0$. Точніше, ми пишемо $f(h)=o(h)$, якщо $f(h)/h\rightarrow0$ при $h\rightarrow 0$. Термін $o(h)$ загалом представляє різні функції $h$ при кожній появі. Так, наприклад, $o(h)+o(h)=o(h)$}, що для малих позитивних $h$,
\end{enumerate}
\begin{equation}\label{11.1}
  \centering
  \begin{gathered}
  \mathbb{P}(N_{t+h}=n+1\big|N_t=n)=\lambda h+o(h),\\
  \mathbb{P}(N_{t+h}=n\big|N_t=n)1-\lambda h +o(h).
  \end{gathered}
\end{equation}

Умова $E$ заслуговує на обговорення. Вона передбачає, що ймовірність того, що виклик надходить за деякий короткий інтервал часу $(t,t+h]$ є приблизно лінійною функцією $h$, і що це наближення стає все кращим і кращим, оскільки $h$ стає все меншим і меншим. З (\ref{11.1}) випливає, що ймовірність двох або більше викликів в інтервалі $(t,t+h]$ задовольняє


\begin{gather*}
    \mathbb{P}(N_{t+h}\geq n+2\big|N_t=n)=1-\mathbb{P}(N_{t+h} \text{ дорівнює } n \text{ або } n+1\big|N_t=n)\\
    =1-[\lambda h+o(h)]-[1-\lambda h+o(h)]\\
    =o(h),
\end{gather*}
так що єдині дві можливі події зі значною ймовірністю (тобто з ймовірністю більше, ніж $h(o)$) передбачає або жоден виклик, що надходить у $(t,t+h]$, або рівно один виклик, що надходить у цей проміжок часу.

Це наша модель надходження телефонних дзвінків. Це примітивна модель, заснована на ідеї випадкових надходжень і отримана за допомогою різних спрощуючих припущень. Для причини, яка незабаром буде зрозуміла, цей випадковий процес $N=(N_t:t\geq 0)$ називається пуассонівським процесом зі швидкістю $\lambda$. Процеси Пуассона можна використовувати для моделювання багатьох явищ, наприклад:
\begin{enumerate}
  \item прихід покупців в магазин,
  \item клацання, випромінювані лічильником Гейгера, коли він записує виявлення радіоактивних частинок,
  \item частота смертей у невеликому місті з відносно стальним населенням (нехтуючи сезонні коливання).
\end{enumerate}

Процеси Пуассона забезпечує винятково хорошу модель випромінювання радіоактивних речовин частинок, коли джерело має довгий період напіврозпаду та повільно розпадається.

Ми можемо представляти результати процесу Пуассона $N$ за допомогою графіка $N_t$ від $t$ (див. малюнок ~\ref{pic1}). Нехай $T$ буде часом, коли $i$-й виклик надходить, так що
\begin{equation}\label{11.2}
  T_i=inf\{t:N_t=i\}.
\end{equation}

Тоді $T_0=0$, $T_0\leq T_1 T_2\leq \dots$ і $N_t=i$, якщо $t$ лежить в інтервалі $[T_i, T_i+1)$. Зазначаємо, що $T_0, T_1, T_2, \dots$ є повністю: якщо ми знаємо $T_i$, то $N_t$ визначається як
\[N_t=max\{n:T_n\leq t\}.\]

Послідовність $T_i$ можна розглядати як <<зворотний процес>> $N$.

\begin{figure}
  \centering
  \includegraphics{picture}
  \caption{Ескіз процесу Пуассона}\label{pic1}
\end{figure}

Умова 1-5 є нашими постулатами для пуассоністького процесу $N$. У наступних двох розділах ми представимо деякі наслідки цих постулатів, відповідаючи на такі питання, як
\begin{enumerate}
  \item якою є функція маси $N_t$ для даного значення $t$?
  \item що можна сказати про розподіл послідовності $T_0, T_1, \dots$ часів, коли надходили виклики?
\end{enumerate}

\begin{exercise}
Якщо $N$ є пуассоністьким процесом зі швидкістю $\lambda$, покажіть, що
$$\mathbb{P}(N_{t+h}=0)=\big[1-\lambda h +o(h)\big]\mathbb{P}(N_t=0)$$
для малих додатких значень $h$. Отже, покажіть, що $p(t)=\mathbb{P}(N_t=0)$ задовольняє диференціальне рівняння.
$$p'(t)=-\lambda p(t).$$
Розв'яжіть це рівняння, щоб знайти $p(t)$.
\end{exercise}

\begin{exercise}
  \textbf{Розрідження} Припусти, що надходження телефонних \\дзвінків на станцію є Пуассонівським процесом $N=(N_t:t\geq0)$ зі швидкістю $\lambda$, і припустимо, що обладнання несправне, так що кожен вхідний дзвінок не записується з імовірністю $q$ (незалежно від усіх інших дзвінків). Якщо $N'$ кількість дзвінків, записаних до часу $t$, показати, що $N'=(N_{t}':t\geq 0)$ є пуассонівським процесом зі швидкістю $\lambda(1-q)$.
\end{exercise}

\begin{exercise}\label{exercise.11.5}
  \textbf{Суперпозиція} Для незалежних потоки телефонних дзвінків надходять на станцію, перший є процесом Пуассона зі швидкістю $\lambda$, а другий є процесом Пуассона зі швидкістю $\mu$. Показати, що об'єднаний потік викликів є процесом зі швидкістю $\lambda + \mu$
\end{exercise}

\subsection{Процес Пуассона}

Пуассонівський процес зі швидкістю $\lambda$ є випадковим процем, який задовольняє постулатам (1-5) розділу (\ref{sub.11.1}). Наш перший результат встановив зв'язок із розподілом Пуассона.

\begin{theorem}\label{theorem.11.6}
  Для кожного $t>0$ випадкова величина $N_t$ має розподіл Пуассона з параметром $\lambda t$. Тобто $t>0$
  \begin{equation}\label{11.7}
    \mathbb{P}(N_t=k)=\frac{1}{k!}(\lambda t)^{k}e^{-\lambda t} \quad \text{для } k = 0, 1, 2, \dots
  \end{equation}
\end{theorem}

З (\ref{11.7}) випливає, що середнє значення та дисперсія $N_t$ зростають лінійно по $t$ зі збільшенням $t$:
\begin{equation}\label{11.8}
  \mathbb{E}(N_t)=\lambda t, \quad \text{var}(N_t)=\lambda t \quad \text{для } t>0
\end{equation}

\begin{proof}
  Так само, як ми встановлюємо різницеві рівняння для процесі дискретним часом, ми встановлюємо тут <<диференціально-різницеві>> рівняння. Дозволяє
  \[p_k(t)=\mathbb{P}(N_t=k)\]
  Зафіксуємо $t\geq0$ і нехай $h$ буде малим позитивним. Основний крок полягає в тому, щоб виразити $N_{t+h}$ через $N_t$ як випливає. Ми використовуємо теорему про розбиття, Теорема 1.48, щоб побачити, що якщо $k\geq 1$,

  \begin{equation}\label{11.9}
  \begin{gathered}
    \mathbb{P}(N_{t+h}=k)=\sum_{i=0}^{k}\mathbb{P}(N_{t+h}=k\big| N_t=i)\mathbb{P}(N_t=i)\\
    =\mathbb{P}(N_{t+h}=k\big|N_t=k-1)\mathbb{P}(N_t=k-1)\\
    +\mathbb{P}(N_{t+h}=k\big|N_t=k)\mathbb{P}(N_t=k)+o(h) \quad \text{за (\ref{11.1})}\\
    =\left[\lambda h + o(h)\right]\mathbb{P}(N_t=k-1)+\\
    \left[1-\lambda h + o(h)\right]\mathbb{P}(N_{t}=k)+o(h) \quad \text{за (\ref{11.1})}\\
    = \lambda h\mathbb{P}(N_t=k-1)+(1-\lambda h)\mathbb{P}(N_t=k)+o(h)
  \end{gathered}
\end{equation}
Отримуємо, що
\begin{equation}\label{11.10}
  p_k(t+h)-p_k(t)=\lambda h\big[p_{k-1}(t) - p_t(t)\big]+o(h).
\end{equation}
дійсно для $k = 1, 2, \dots$. Ми ділимо обидві частини (\ref{11.10}) на $h$ і беремо межу як $h\downarrow 0$ та отримаємо
\begin{equation}\label{11.11}
  p_{k}'(t)=\lambda p_{k-1}(t)-\lambda p_k(t) \quad \text{для } k = 1,2, \dots,
\end{equation}
де $p_k'(t)$ є похідною від $p_k(t)$ по $t$. При $k=0$ (\ref{11.9}) набуває вигляду
\begin{gather*}
  \mathbb{P}(N_{t+h}=0)=\mathbb{P}(N_{t+h}=0\big|N_t=0)\mathbb{P}(N_t=0) \\
  =(1-\lambda h)\mathbb{P}(N_t=0)+o(h),
\end{gather*}
дає таким чином
\begin{equation}\label{11.12}
  p_0'(t)=-\lambda p_0(t).
\end{equation}

Рівняння (\ref{11.11}) і (\ref{11.12}) є системою диференціально-різницею рівнянь для функції $p_0(t), p_1, \dots$, і ми хочемо їх розв'язати з урахуванням граничної умови $N_0=0$, що еквівалентно умові
\begin{equation}\label{11.13}
  p_k(0)=\begin{cases}
             1 \quad \text{якщо } k=0, \\
             0 \quad \text{якщо } k\neq 0.
           \end{cases}
\end{equation}

Ми розглянемо 2 методи розв'язання цієї системи рівнянь.

\textbf{Розв'язок А (за індукцією)} Рівняння (\ref{11.12}) містить лише $p_0(t)$. Його загальне рішення таке $p_0(t)=Ae^{-\lambda t}$, а довільна стала $A$ знаходиться з (\ref{11.13}) рівно 1. Отже
\begin{equation}\label{11.14}
  p_o(t)=e^{-\lambda t} \quad \text{для } t\geq 0.
\end{equation}

Підставте це в (\ref{11.11}) з $n=1$, щоб отримати
$$p_1'(t)+\lambda p_1(t)=\lambda e^{-\lambda t}$$
який за допомогою інтегруючого фактора та граничної умови дає
\begin{equation}\label{11.15}
  p_1(t)=\lambda te^{-\lambda t}  \quad \text{для } t\geq 0.
\end{equation}
Продовжуйте таким чином, щоб знайти це
$$p_2(t)=\frac{1}{2}(\lambda t)^2e^{\lambda t}.$$
Тепер вгадайте загальний розв'язок (\ref{11.7}) і доведіть його з (\ref{11.11}) індукцією.

\textbf{Розв'язок Б (генеруючи функції)} Цей метод кращий і має інші застосування, ми використовуємо функцію, що створює ймовірність $N_t$, а саме
$$G(s,t)=\mathbb{E}(s^{N_t})=\sum_{k=0}^{\infty}p_k(t)s^k.$$

Помножимо обидві сторони (\ref{11.11}) на $s^k$ та додамо за значеннями $k=1, 2, \dots$, щоб отримати це
$$\sum_{k=1}^{\infty}p_k'(t)s^k=\lambda\sum_{k=1}^{\infty}p_{k-1}(t)s^k-\lambda\sum_{k=1}^{\infty}p_k(t)s^k.$$
Додамо (\ref{11.12}) до цього об'єктивним чином та зазначимо, що
$$\sum_{k=1}^{\infty}p_{k-1}(t)s^k=sG(s,t)$$
і (плюс або мінус дещиця математичної строгості)\
$$\sum_{k=0}^{\infty}p_k'(t)s^k=\frac{\partial G}{\partial t},$$
отримаємо
\begin{equation}\label{11.16}
  \frac{\partial G}{\partial t}=\lambda s G - \lambda G.
\end{equation}
диференціальне рівняння з граничною умовою
\begin{equation}\label{11.17}
  G(s, 0) = \sum_{k=0}^{\infty}p_k'(0)s^k = 1 \quad \text{за (\ref{11.13})}
\end{equation}

Рівняння (\ref{11.16}) може бути записаним у вигляді
$$\frac{1}{G}\frac{\partial G}{\partial t}=\lambda(s-1).$$

Це нагадує частково диференціальне рівняння, але для кожного заданого значення $s$ воно може бути інтегрованим звичайним способом щодо $t$, що
$$\log G = \lambda t(s-1)+A(s),$$
де $A(s)$ --- довільна функція $s$. Використовуйте (\ref{11.17}), щоб знайти, що $A(s)=0$ для всіх $s$, отже
$$G(s, t)=e^{-\lambda t(s-1)}=\sum_{k=0}^{\infty}\Bigg(\frac{1}{k!}(\lambda t)^ke^{-\lambda t}\Bigg)s^k.$$

Зчитавши коефіцієнт $s_k$, ми отримали що
$$p_k(t)=\frac{1}{k!}(\lambda t)^ke^{-\lambda t}$$
як вимагалось
\end{proof}

\begin{exercise}
  Якщо $N$ --- процес Пуассона зі швидкістю $\lambda$, покажіть, що $var(N_t/t)\rightarrow 0$ при $t \rightarrow \infty$.
\end{exercise}

\begin{exercise}
  Якщо $N$ є пуассонівським процесом зі швидкістю $\lambda$, покажіть, що для $t>0$
  \begin{gather*}
    \mathbb{P}(N_t \text{ є парним})=e^{-\lambda t} \cosh \lambda t, \\
    \mathbb{P}(N_t \text{ є не парним})=e^{-\lambda t} \sinh \lambda t,
  \end{gather*}
\end{exercise}

\begin{exercise}\label{exercise.11.20}
  Якщо $N$ є пуассонівським процесом зі швидкістю $\lambda$, покажіть, що функція, що створює момент
  $$U_t=\frac{N_t-\mathbb{E}(N_t)}{\sqrt{\text{var}(N_t)}}$$
  є
  $$M_t(x)=\mathbb{E}(e^{xU_t})=\exp\big[-x\sqrt{\lambda t}+\lambda t(e^{x/\sqrt{\lambda t}} -1)\big],$$

  Вивести, якщо $t\rightarrow \infty$,
  $$\mathbb{P}(U_t\leq u) \rightarrow \int_{-\infty}^{u}\frac{1}{\sqrt{2\pi}}e^{-\frac{1}{2}v^2}\mathrm{d}v \quad \text{для} \quad u \in \mathbb{R}$$
  Це центральна гранична теорема для процесу Пуассона.
\end{exercise}

\subsection{Час між прибуттями та експоненціальний розподіл}
Нехай $N$ --- процес Пуассона зі швидкістю $\lambda$. Часи прибуття $T_0, T_1, \dots$, $N$ визначаються як раніше за $T_0=0$ і
\begin{equation}\label{11.21}
  T_i=\text{inf}\{t: N_t=i\} \quad \text{для $i = 1, 2, \dots.$}
\end{equation}

Іншими словами, $T_i$ --- час надходження $i$-го телефонного дзвінка. Час між прибуттями $X_1, X_2, \dots$ --- час між послідовними прибуттями,
\begin{equation}\label{11.22}
  X_i=T_i-T_{i-1} \quad \text{для $i = 1, 2, \dots.$}
\end{equation}

Розподіл $X_i$ дуже просто описати.

\begin{theorem}\label{theorem.11.23}
  У пуассонівському процесі зі швидкістю $\lambda$ час між надходженнями $X_1, X_2, \dots$ дорівнює незалежним випадковим величинам, кожна з яких має експоненціальний розподіл із параметром $\lambda$.
\end{theorem}

Цей результат демонструє тісний зв'язок між постулатами для процесу Пуассона і експоненціальним розподілом. Теорема (\ref{theorem.11.23}) є лише вершиною айсберга: глибше дослідження випадкових процесів безперервного часу показує, що експоненціальний розподіл є наріжним каменем для процесів, які задовольняють умову незалежності, таку як припущення $D$. Причиною цього є те, що експоненціальний розподіл є єдиним неперервним розподілом з так званої властивості відсутності пам'яті.

\begin{definition}\label{def.11.24}
    Кажуть, що позитивна випадкова змінна $X$ має властивість відсутність пам'яті, якщо
    \begin{equation}\label{11.25}
      \mathbb{P}\big(X>u+v\big|X>u\big)=\mathbb{P}(X>v), \quad \text{якщо $u, v \geq 0$.}
    \end{equation}
\end{definition}

Думаючи про $X$ як про час, який минув до якоїсь події $A$, скажімо, умови (\ref{11.25}) вимагає, що $A$ не відбулося до часу $u$, то час, який мине згодом (між $u$ і появою $A$) не залежить від значення $u$: <<випадкова величина $X$ не пам'ятає, скільки їй років, коли планує своє майбутнє>>.

\begin{theorem}\label{theorem.11.26}
  Безперервна випадкова змінна $X$ має властивість відсутності пам'яті, тоді й тільки тоді, коли $X$ має експоненціальний розподіл.
\end{theorem}

\begin{proof}
  Якщо $X$ має експоненціальний розподіл із параметром $\lambda$, то для $u, v \geq0$,  
  \begin{eqnarray*}
    \mathbb{P}\bigg(X>u+v\bigg|X > u\bigg)= \frac{\mathbb{P}(X>u+v\text{ і }X>u)}{\mathbb{P}(X>u)} \\
    =\frac{\mathbb{P}(X>u+v)}{\mathbb{P}(X>u)} \quad \text{Оскільки } u \leq u+v \\
    =\frac{e^{-\lambda(u+v)}}{e^{-\lambda u}} \quad \text{з прикладу (5.22)} \\
    = e^{-\lambda v}=\mathbb{P}(X>v),
  \end{eqnarray*}
так що $X$ має властивість відсутності пам'яті.

Навпаки, припустимо, що $X$ є додатним і неперервним і має властивість відсутності пам'яті. Нехай $G(u)=\mathbb{P}(X>u)$ для $u\geq 0$. Ліва частина (\ref{11.25}) є
$$\mathbb{P}\bigg(X>u+v\bigg|X>u\bigg)= \frac{\mathbb{P}(X>u+v)}{\mathbb{P}(X>u)}=\frac{G(u+v)}{G(u)},$$
так що $G$ задовольнятиме <<рівняння функції>>
\begin{equation}\label{11.27}
  G(u+v)=G(u)G(v) \quad \text{для } u, v\geq0.
\end{equation}

Функція $G(u)$ не зростає в дійсній змінній $u$, а всі ненульові не зростають розв'язки (\ref{11.27}) мають вигляд
\begin{equation}\label{11.28}
  G(u)=e^{-\lambda u} \quad u \geq 0,
\end{equation}
де $\lambda$ деяка константа. Виведення (\ref{11.28}) є цікавою вправою в аналізі (\ref{11.27}), і ми пропонуємо читачеві перевірити це. Спочатку використовуйте (\ref{11.27}), щоб показати, що $G(n)=G(1)^n$ для $n=0,1,2,\dots$, тоді виведіть, що $G(u)=G(1)^u$ для всіх невід'ємних раціональних чисел $u$, і, нарешті, використовуйте монотонність, щоб розширити це від раціональних до дійсних чисел.
\end{proof}

\textbf{Ескізне доведення теореми (\ref{theorem.11.23}).} Спочатку розглянемо $X_1$. Очевидно,
$$\mathbb{P}(X_1>u)=\mathbb{P}(N_u=0) \quad \text{для } u \geq 0,$$
і теорема (\ref{theorem.11.6}) дає
$$\mathbb{P}(X_1>u)=e^{-\lambda u} \quad \text{для } u \geq 0,$$

так що $X_1$ має експоненціальний розподіл з параметром $\lambda$. Від припущення незалежності $D$, надходження в інтервалі $(0, X_1]$ не залежить від надходжень, наступних за $X_1$, з цього випливає, що <<час очікування>> $X_2$ для наступного прибуття після $X_1$ не залежить від $X_1$. Більш того, надходження відбувається <<однорідно>> за часом (оскільки ймовірність прибуття в $(t, t+h]$ не залежить від $t$, а лише від $h$ --- запам'ятайте (\ref{11.1})), враховуючи, що $X_2$ має такий же розподіл, як $X_1$. Подібним чином, усі $X_i$ є незалежними з тим самим розподілом, що й $X_1$.

Аргумент наведеного вище доказу є неповним, оскільки крок передбачає незалежність маємо справу з інтервалом $(0, X_1)$ \emph{випадкової} довжини. Це не зовсім тривіальне завдання зробити цей крок, і саме з цієї причини доказ тут лише нарисовано. Необхідна частина Пуассонівсього процесу іноді називають <<сильною марковською властивістю>>, і ми повернемось для процесів у дискретному, а не безперервному часі в розподілі 12.7.

Вище ми показали, що якщо $N$ є пуассонівським процесом з параметром $\lambda$, часи $X_1, X_2, \dots$ між надходженнями в цьому процесі є незалежними та однаково розподіленими експоненціальним розподілом, параметр $\lambda$. Цей висновок характеризує процес Пуассона, в тому сенсі, що процеси Пуассона є єдиними <<процесами надходження>> з цією властивістю. Більш правильно, маємо наступне. Нехай $X_{1}^{*}, X_{2}^{*}, \dots$ є незалежними випадковими величинами, кожна з яких має експоненціальний розподіл з параметром $\lambda (>0)$, і припустимо, що телефон у Гранд Готелі замінено на дуже особливу нову модель, яка запрограмована на дзвінок у певний час
$$T_{1}^{*}=X_{1}^{*},\quad T_{2}^{*}=X_{1}^{*}+X_{2}^{*}, \quad T_{3}^{*}=X_{1}^{*}+X_{2}^{*}+X_{3}^{*}, \quad \dots,$$
так що час, який мине між $(i-1)$-м та $i$-м викликами дорівнює $X_{i}^{*}$. Дозволяє
$$N_{t}^{*}=\max\{k: T_{k}^{*}\leq t\}$$
--- кількість дзвінків, що надійшли до час $t$. Тоді процес $N^{*}=(N_{t}^{*}:t\geq 0)$ є процес Пуассона зі швидкістю $\lambda$, так що, з точки зору Білла, новий телефон поводиться точно так само (статистично кажучи), як і старі моделі.

\begin{example}
    Припустимо, що автобуси до центру міста прибувають на зупинку на розі у спосіб процесу Пуассона. Знаючи це, Девід очікує, що він буде чекати експоненціально розподілений проміжок часу до того, як його забере автобус. Якщо він прийде на автобусну зупинку, і Доріс розповість йому що вона чекає вже 50 хвилин, то це не хороша і не погана новина для нього, оскільки експоненціальний розподіл має властивість відсутності пам'яті. Подібним чином, якщо він підійде як раз вчасно для того щоб побачити автобус, який відпраляється, то йому не потрібно хвилюватися, що він чекатиме довше, ніж зазвичай. Ці властивості є характеристиками процесу Пуассона.
\end{example}


\begin{problem}
     Нехай, $M$ та $N$ --- незалежні пуассонівські процеси, $M$ має швидкість $\lambda$, а $N$ --- швидкість $\mu$. Використайте результат задачі 6.9.4, щоб показати, що процес $M + N = (M_t + N_t : t \geq 0)$ є процесом Пуассона зі швидкістю $\lambda + \mu$. Порівняйте цей метод із методом вправи (\ref{exercise.11.5}).
\end{problem}

\begin{problem}
  Якщо $T_i$ є часом $i$-го надходження в пуассоніський процес $N$, то покажіть, що $N_t < k$ тоді і тільки тоді, коли $T_k > t$. Використовуйте теорему (\ref{theorem.11.23}) і центральну граничну теорему, теорему 8.25, щоб вивести, що $t \rightarrow \infty$,
  $$ \mathbb{P}(\frac{N_t-\lambda t}{\sqrt{\lambda t}}\leq u)\rightarrow \int_{-\infty}^{u}\frac{1}{\sqrt{2\pi}}e^{-\frac{1}{2}v^2}\, \mathrm{d}v \quad \text{для $u \in \mathbb{R}.$} $$
  Порівняйте це з задачею (\ref{exercise.11.20}).
\end{problem}

\begin{problem}
  Дзвінки надходять на телефонну станцію у вигляді процесу Пуассона зі швидкістю $\lambda$, але оператор часто відволікається і відповідає лише на кожен другий дзвінок. Який загальний інтервал часу, що проходить між послідовними викликами, які викликають відповідь?
\end{problem}

\end{document} 