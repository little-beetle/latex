%\documentclass[14pt]{extarticle}


\documentclass[fontsize=14pt]{scrartcl}

% пакети
\usepackage[cp1251]{inputenc}
\usepackage{latexsym,amsmath,amsfonts,amsthm,amscd,amssymb,lscape,ifthen}
\usepackage[pdftex]{graphicx}
\DeclareGraphicsExtensions{.pdf,.png,.jpg,.bmp}
\RequirePackage{caption}
\usepackage{float}
\usepackage{wrapfig}
\usepackage{geometry}
\usepackage{tikz}
\usepackage{caption}
%\usepackage{venndiagram}
\usepackage[colorlinks,linkcolor=black]{hyperref}
\usepackage{secdot}



\usepackage{indentfirst}
\usepackage{geometry} % Меняем поля страницы
%\geometry{left=1.05cm}% левое поле
%\geometry{right=1.05cm}% правое поле
%\geometry{top=1cm}% верхнее поле
%\geometry{bottom=1.5cm}% нижнее поле
\setlength{\textwidth}{170mm}
\setlength{\textheight}{235mm}
%\setlength{\topmargin}{-30mm}
%\setlength{\oddsidemargin}{-5mm}
%\voffset=-10mm
\hoffset=-5mm

\pagestyle{plain}
%\usepackage[utf8]{inputenc}
\usepackage[cp1251]{inputenc}
\usepackage[english,ukrainian]{babel}
\usepackage{amsmath}
\usepackage{pgfplots}
\usepackage{multicol}
\usepackage{tikz}
\usepackage{amssymb}
\usepackage{array}
\usepackage{longtable}
\usepackage{fancyhdr}
\usepackage{xspace,amscd,graphicx,t1enc}

\renewcommand{\labelenumi}{\theenumi)}

\newtheorem{exm}{Приклад}[section]
\theoremstyle{theorem}
\newtheorem{thm}{Теорема}[section]
\newtheorem{ozn}{Означення}[section]
\theoremstyle{proof}
\newtheorem*{dov}{Доведення}

\sloppy
%\pagestyle{headings}


\tikzset{global scale/.style={
		scale=#1,
		every node/.style={scale=#1}
	}
}



\interfootnotelinepenalty=10000


\begin{document}
\begin{thm}\label{13.4.1}
Ізольована особлива точка $a\in\mathbb{C}$ функції $f(z)$ є полюсом тоді і тільки тоді, коли головна частина ряду Лорана в околі точки $a$ містить лише скінченне число відмінних від нуля членів.
\begin{equation}\label{13.4.4}
  f(z)=\sum_{k=-N}^{\infty}c_k(z-a)^k, \quad N>0
\end{equation}
\end{thm}
\begin{dov}
\textbf{Необхідність.} Нехай $a$ --- полюс; оскільки $\lim_{z \to a}f(z)=\infty$, то існує проколотий окіл точки $a$, де $f(z)$ регулярна і відмінна від нуля. В цьому околі регулярна функція $\varphi(z)=\frac{1}{f(z)}$, причому існує $\lim_{z \to a}\varphi(z)=0$. Значить $a$ є усувною точкою (нулем) функції $\varphi$ і в нашому околі справедливий розклад
$$\varphi(z)=b_N(z-a)^N+b_{N+1}(z-1)^{N+1}+\dots, \quad (b_N\neq 0) $$

Але тоді в тому ж околі ми маємо
\begin{equation}\label{13.4.5}
  f(z)=\frac{1}{\varphi (z)}=\frac{1}{(z-a)^N}\cdot\frac{1}{b_N+b_{N+1}(z-a)+\dots}
\end{equation}
при цьому другий множник є функцією ругулярною в точці $a$, а значить має розклад в ряд Тейлора
$$\frac{1}{b_N+b_{N+1}(z-1)+\dots}=c_{-N}+c_{-N+1}(z-a)+\dots \quad (c_{-N}=\frac{1}{b_N}\neq 0)$$

Підставивши цей розклад в (\ref{13.4.5}), будемо мати
$$f(z)=\frac{c_{-N}}{(z-a)^N}+\frac{c_{-N+1}}{(z-a)^{N-1}}+\dots +\sum_{k=0}^{\infty}c_k(z-a)^k.$$
це є розклад в ряд Лорана функції $f(z)$ в проколотому околі точки $a$, і ми бачимо, що його головна частина містить скінченне число членів.

\textbf{Достатність.} Нехай $f(z)$ в проколотому околі точки $a$ зображаються розкладом в ряд Лорана (\ref{13.4.4}), головна частина якого містить скінченне число членів; нехай ще $c_{-N}\neq 0$. Тоді $f(z)$ регулярна в цьому околі, як функція $\varphi(z)=(z-a)^Nf(z)$. Ця функція в даному околі зображається рядом
$$\varphi(z)=c_{-N}+c_{-N+1}(z-a)+\dots,$$
звідки видно, що $a$ є усувною точкою і існує $\lim_{z \to a}\varphi(z)=c_{-N}\neq 0$. Але тоді функція $f(z)=\frac{\varphi (z)}{(z-a)^N}$ прямує до $\infty$ при $z \to a$, тобто точка $a$ є полюсом
\end{dov}

Відмітимо ще один факт прозв'язок полюсів з нулями.

\begin{thm}\label{thm.13.4.1}
  Точка $a$ є полюсом функції $f(z)$ в тому і тільки тому випадку, коли функція $\varphi(z)=\frac{1}{f(z)}$, $\varphi(z) \not\equiv 0$, регулярна в околі точки $a$ і $\varphi(a)=0$.
\end{thm}

\begin{dov}
Необхідність умови доведена при доведенні теореми (\ref{thm.13.4.1}). Доведемо її достатність. Якщо $\varphi \not\equiv 0$ регулярна в точці $a$ і $\varphi(z)$, то за теоремою єдності (п. 12.3) існує проколотий окіл цієї точки, в якому $\varphi(z)\neq 0$. В цьому околі функція $f(z)=\frac{1}{\varphi(z)}$ регулярна, і, знаючи, $a$ є ізольованоюю особливою точкою $f(z)$. Але $\lim_{z \to a}f(z)=\infty$, значить $a$ є полюсом.
\end{dov}
Цей зв'язок дозволяє сформулюювати.

\begin{ozn}
  Порядком полюса в точці $a$ функції $f(z)$ називається порядок цієї точки як нуля функції $\varphi(z)=\frac{1}{f(z)}$.
\end{ozn}

З доведення теореми (\ref{thm.13.4.1}) видно, що порядок полюса співпадає з номером $N$ старшого члена головної частини розкладу функції $f(z)$ в ряд Лорана в проколотому околі полюса.

\begin{exm}
Для функції $f(z)=\frac{1}{\sin\frac{1}{z}}$ точки $z_k=\frac{1}{k\pi}$, $k=\pm 1, \pm 2, \dots,$, є полюсами першого порядку, оскільки функція $g(z)=\frac{1}{f(z)}=\sin\frac{1}{z}$ регулярна при $z\neq 0$, а точки $z_k$ є її нулями першого порядку,($g'(z_k)\neq 0$). Значить, точки $z=0$ є неізольованою особливою точкою. Точка $z=\infty$ --- першого порядку для $f(z)$, бо $f(z)\sim z$ ($\lim_{z \to \infty}\frac{f(z)}{z}=1$) при $z \to \infty$.
\end{exm}

\begin{exm}
Точка $z=0$ є полюсом першого порядку для функції $f(z)=\frac{1-\cos z}{(e^z-1)^3}$. Точки $z_k=2k\pi i (k=\pm 1, \pm 2,\dots)$ полюси третього порядку для $f(z)$ оскільки ці точки є нулями третього порядку для функції $f(z)=(e^z-1)^3$, а $1-\cos z \neq 0$.
\end{exm}

\subsection{Характеристична властивість істотно особливої точки}\label{13.5}

\begin{thm}\label{thm.13.5.1}
  Ізольована особлива точка $a$ функції $f(z)$ є суттєво особливою тоді і тільки тоді, коли головна частина розкладу Лорана $f(z)$ в околі точки $a$ містить нескінченну кількість відмінних від нуля членів.
\end{thm}

\begin{dov}
Доведення цієї теореми по суті міститься в теоремах \ref{thm.13.4.1} пунктів 13.3 і 13.4. Бо якщо головна частина містить нескінченне число членів, то $a$ не може бути ні усувною точкою, а ні полюсом; якщо $a$ --- суттєво особлива точка, то головна частина не може бути відсутньою, і не може містити скінченне число членів.
\end{dov}

\begin{thm}\label{thm.13.5.2}
  \textbf{(Ю. В. Сохоцкий).} Якщо $a$ є суттєво особливою точкою функції $f$, то для довільного числа $A\in \bar{\mathbb{C}}$ можна знайти послідовність точок $z_k \to a$ таку, що
  $$\lim_{k\to\infty}f(z_k)=A.$$
  Цю теорему можна сформулювати ще так: в як завгодно малому околі суттєво особливої точки функція $f(z)$ приймає значення як завгодно близькі до довільного наперед заданого числа, скінченного чи нескіченного.
\end{thm}

\begin{dov}
Нехай $A=\infty$. Покажемо, що існує послідовність точок $z_k$, \\ $\lim_{k\to\infty}z_k=a$, таких, що $\lim_{z\to a}f(z_k)=\infty$. Позначимо для скорочення через $P(z-a)$ правильну частину розкладу Лорана (див п. 13.2, формула (1)-(3)), яка містить додатні степені (z-a) і вільний член, $a$ через $Q(\frac{1}{z-a})$ його головну частину, що містить від'ємні степені $z-a$. Тому формулу 1 п. 13.2 можемо переписати у вигляді:
\begin{equation}\label{13.5.1}
  f(z)=P(z-a)+Q(\frac{1}{z-a}).
\end{equation}
Що стосується правильної частини $P(z-a)$, що при $z\to a$ маємо
\begin{equation}\label{13.5.2}
  \lim_{z\to a}P(z-a)=c_0.
\end{equation}

Покладаючи $\frac{1}{z-a}=z'$ в головній частині $Q(\frac{1}{z-a})$, будемо мати
\begin{equation}\label{13.5.3}
  Q(\frac{1}{z-a})=Q(z')=c_{-1}z'+c_{-2}(z')^2+\dots+c_{-k}(z')^k+\dots.
\end{equation}

Оскільки ряд $Q(\frac{1}{z-a})$ збігається скрізь, крім точки $z=a$ (п. 13.2), то ряд (\ref{13.5.3}), очевидно, буде збіжним у всі площині комплексного змінного $z'$. Функція $Q(z')$ не може бути обмеженою у всій площині комплексного змінного $z'$. (це за теоремою Ліувіля: якщо функція $f(z)$ регулярна у всій площині, є обмеженою по модулю, то вона є тотожньою сталою). Таким чином $\forall n \in \mathbb{N}$, знайдеться точка $z'_n$, $|z'_n|>n$, така, що будемо мати $|Q(z'_n|>n$. Заставляючи $n$ пробігати значення $1,2,3,\dots,k,\dots$, одержимо послідовність точки $z'_1,z'_2,\dots,z'_k,\dots$, яка прямує до $\infty$ і таку, що будемо мати
$$\lim_{z'_k\to\infty}Q(z'_k)=\infty$$

Повертаючись до попереднього змінного $z$ бачимо на основі рівності $\frac{1}{z-a}=z'$, що послідовність точок $z'_k$ перетворюється в послідовність точок $z_1, z_2, \dots, z_k, dots$ збіжну до точки $a$, таку, що маємо
\begin{equation}\label{13.5.4}
  \lim_{z_k\to a}Q(\frac{1}{z_k-a})=\infty.
\end{equation}

Заставляючи точки $z\to a$, проходим послідовність точок $z_k$, бачимо з рівності (\ref{13.5.1}) на основі (\ref{13.5.2}) і (\ref{13.5.4}):
$$\lim_{z_k\to a}f(z_k)=\infty.$$

Нехай тепер $A$ є довільне скінченне комплексне число. Може трапитись, що в довільно малому околі точки $a$ існує точка $z$, така, що маємо $f(z)=A$. У цьому випадку теорема Сохоцького справедлива. Таким чином, можна припустити, що в достатньо малому околі точки $a$ функція $f(z)$ не рівна $A$. Якщо так, то функція $\varphi(z)=\frac{1}{f(z)-A}$ буде регулярною в цьому околі точки $a$, крім точки $a$, яку вона має в якості суттєво особливої точки (тому, що $z=a$ --- суттєво особлива точки для $f(z)-A$). За доведеним $\exists$ послідовність точок $\{z_n\}$ збіжна до точки $a$, така, що $\lim_{z_n \to a}\varphi(z_n)=\infty$, звідси слідує, що $\lim_{z_n\to a}f(z_n)=A.$
\end{dov}

\begin{exm}\label{exm.13.5.1}
  1. Функція $e^{\frac{1}{z}}$ має при $z=0$ суттєво особливу точку. Розклад Лорана в околі цієї точки буде
  $$e^{\frac{1}{z}}=\sum_{k=0}^{\infty}\frac{1}{k!z^k}$$

  2. Для функції $f(z)=\cos z$ точки $z=\infty$ є суттєво особливою, бо головна частина $f_1(z)$ ряду Лорана $f(z)$ в околі точки $z=\infty$ містить нескінченну кількість членів.
  $$f_1(z)=\sum_{k=1}^{\infty}(-1)^k\frac{z^{2k}}{(2k)!}$$
\end{exm}

Більш глибоким твердженням, ніж теорема Сохоцьокого, є така.
\begin{thm}
  \textbf{Пікара.} В довільному околі суттєво особливої точки функція приймає, і притому нескінченне число разів, довільне значення, крім, можливо, одного.
\end{thm}

\begin{exm}\label{exm.13.5.2}
  Для функції $f(z)=e^z$ точка $z=\infty$ є суттєво особливою (див. приклад \ref{exm.13.5.1}). Розглянемо рівняння
  \begin{equation}\label{13.5.5}
    e^z=A, \quad(A\neq 0),
  \end{equation}

  Це рівняння має такі розв'яки
  \begin{equation}\label{13.5.6}
    z_k=\ln|A|+i(\arg A+2k\pi),
  \end{equation}
  де $\arg A$ --- фіксоване значення аргумента числа $A, k=0, \pm 1,\pm 2, \dots,.$. З (\ref{13.5.5}) і (\ref{13.5.6}) слідує, що в довільному околі точки $z=\infty$ є нескічненна множина точок $z_k$ в яких функція $e^z$ приймає значення, рівне $A(A\neq0)$, значення $A$ функція $e^z$ не приймає.
\end{exm}

\begin{exm}
  Точка $z=\infty$ є суттєво особливою для функції $f(z)=\sin z$ і для $\forall A$ рівняння $\sin z = A$ має безліч розв'язків.
  $$z_k=\frac{1}{i}\ln(iA+\sqrt{1-A^2})+2k\pi, \quad k \in \mathbb{Z}$$
\end{exm}

\begin{exm}
  Нехай функції $f(z)$ і $g(z)$ регулярні в точці $a$, $g(z) \not\equiv 0$. Тоді для функції $h(z)=\frac{f(z)}{g(z)}$ точка $a$ є або полюсом, або точкою регулярності.
\end{exm}

\end{document}